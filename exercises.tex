\documentclass[10pt]{article}
\usepackage{bm}
\usepackage{amsmath}
\usepackage{amssymb}
\usepackage[paperheight=26 true cm,paperwidth=18.4 true cm,
top=2.6 true cm,bottom=2.2 true cm,left=1.8 true cm,right=1.8 true cm]{geometry}

\title{Exercises of State Estimation for Robotics}

% put your name here
\author{Xiang Gao}

\begin{document}

\maketitle

\section{Chapter 2}
\subsection{ex 2.1}
\begin{equation}
{\mathbf{u}^T}\mathbf{v} = \sum\limits_{i = 1}^n {{u_i}{v_i}}  = \mathrm{tr}\left( {\mathbf{v}{\mathbf{u}^T}} \right)
\end{equation}

\subsection{ex 2.2}
\begin{equation}
\begin{split}
H\left( {\mathbf{x},\mathbf{y}} \right) &=  - \iint {p\left( {\mathbf{x},\mathbf{y}} \right)\ln }p\left( {\mathbf{x},\mathbf{y}} \right)\mathrm{d}\mathbf{x}\mathrm{d}\mathbf{y} \\ 
&=  - \iint {p\left( \mathbf{x} \right)p\left( \mathbf{y} \right)\left( {\ln p\left( \mathbf{x} \right) + \ln p\left( \mathbf{y} \right)} \right)\mathrm{d}\mathbf{x}\mathrm{d}\mathbf{y}} \\ 
&=  - \int {p\left( \mathbf{x} \right)\ln p\left( \mathbf{x} \right)\underbrace {\left( {\int {p\left( \mathbf{y} \right)\mathrm{d}\mathbf{y}} } \right)}_1\mathrm{d}\mathbf{x} - \int {p\left( \mathbf{y} \right)\ln p\left( \mathbf{y} \right)\underbrace {\left( {\int {p\left( \mathbf{x} \right)\mathrm{d}\mathbf{x}} } \right)}_1\mathrm{d}\mathbf{y}} }  \\ 
&= H\left( \mathbf{x} \right) + H\left( \mathbf{y} \right) \\ 
\end{split}
\end{equation}

\subsection{ex 2.3}
\begin{equation}
\begin{split}
E\left( {\mathbf{x}{\mathbf{x}^T}} \right) &= E\left( {\left( {\mathbf{x} - \boldsymbol{\mu}  + \boldsymbol{\mu} } \right){{\left( {\mathbf{x} - \boldsymbol{\mu}  + \boldsymbol{\mu} } \right)}^T}} \right) \hfill \\
&= E\left( {\underbrace {\left( {\mathbf{x} - \boldsymbol{\mu} } \right){{\left( {\mathbf{x} - \boldsymbol{\mu} } \right)}^T}}_{\boldsymbol{\Sigma}}  + \underbrace {\boldsymbol{\mu} {{\left( {\mathbf{x} - \boldsymbol{\mu} } \right)}^T}}_{\boldsymbol{\mu} {\mathbf{0}^T}} + \underbrace {\left( {\mathbf{x} - \boldsymbol{\mu} } \right){\boldsymbol{\mu} ^T}}_{\mathbf{0}{\boldsymbol{\mu} ^T}} + \boldsymbol{\mu} {\boldsymbol{\mu} ^T}} \right) \hfill \\
&= \boldsymbol{\Sigma}  + \boldsymbol{\mu} {\boldsymbol{\mu} ^T} \hfill \\ 
\end{split}
\end{equation}

\subsubsection{ex 2.4}
The integrate of an odd function is zero in the symmetric interval.
\begin{equation}
	\begin{split}
		E\left(\mathbf{x} \right) &= \int_{-\infty}^{+\infty}\boldsymbol{x}p\left(\boldsymbol{x}\right)d\boldsymbol{x} \hfill \\
		&= \int_{-\infty}^{+\infty}\frac{\boldsymbol{x}}{\sqrt{(2\pi)^N\text{det}\boldsymbol{\Sigma}}}\text{exp}\left(-\frac{1}{2}(\boldsymbol{x}-\boldsymbol{\mu})^T\boldsymbol{\Sigma}^{-1}(\boldsymbol{x}-\boldsymbol{\mu})\right)d\boldsymbol{x} \hfill \\  
		&=  
		\int_{-\infty}^{+\infty}\frac{\boldsymbol{y}+\boldsymbol{\mu}}{\sqrt{(2\pi)^N\text{det}\boldsymbol{\Sigma}}}\text{exp}\left(-\frac{1}{2}\boldsymbol{y}^T\boldsymbol{\Sigma}^{-1}\boldsymbol{y}\right)d\boldsymbol{y} \hfill \\
		&=
		\underbrace{\int_{-\infty}^{+\infty}\frac{\boldsymbol{y}}{\sqrt{(2\pi)^N\text{det}\boldsymbol{\Sigma}}}\text{exp}\left(-\frac{1}{2}\boldsymbol{y}^T\boldsymbol{\Sigma}^{-1}\boldsymbol{y}\right)d\boldsymbol{y}}_{0} +\hfill
		\\& \int_{-\infty}^{+\infty}\frac{\boldsymbol{\mu}}{\sqrt{(2\pi)^N\text{det}\boldsymbol{\Sigma}}}\text{exp}\left(-\frac{1}{2}\boldsymbol{x}^T\boldsymbol{\Sigma}^{-1}\boldsymbol{x}\right)d\boldsymbol{x} \hfill
		\\&=
		\boldsymbol{\mu}\underbrace{\int_{-\infty}^{+\infty}\frac{1}{\sqrt{(2\pi)^N\text{det}\boldsymbol{\Sigma}}}\text{exp}\left(-\frac{1}{2}\boldsymbol{y}^T\boldsymbol{\Sigma}^{-1}\boldsymbol{y}\right)d\boldsymbol{y}}_{\boldsymbol{y}\sim\boldsymbol{N}(0,\boldsymbol{\Sigma})} \hfill
		\\&=
		\boldsymbol{\mu}\underbrace{\int_{-\infty}^{+\infty}p\left(\boldsymbol{y}\right)d\boldsymbol{y}}_{1} \hfill
		\\&=   \boldsymbol{\mu}
	\end{split}
\end{equation}

\end{document}
